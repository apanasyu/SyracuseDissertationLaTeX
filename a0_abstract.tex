\section*{\centering{Abstract}}
\thispagestyle{empty}
Identifying authoritative influencers related to a geographic area (geo-influencers) can aid content recommendation systems and local expert finding. This thesis addresses this important problem using Twitter data.

A geo-influencer is identified via the locations of its followers. On Twitter, due to privacy reasons, the location reported by followers is limited to profile via a textual string or messages with coordinates. However, this textual string is often not possible to geocode and less than 1\% of message traffic provides coordinates. First, the error rates associated with Google's geocoder are studied and a classifier is built that gives a warning for self-reported locations that are likely incorrect. Second, it is shown that city-level geo-influencers can be identified without geocoding by leveraging the power of Google search and follower-followee network structure. Third, we illustrate that the global vs. local influencer, at the timezone level, can be identified using a classifier using the temporal features of the followers. For global influencers, spatiotemporal analysis helps understand the evolution of their popularity over time. When applied over message traffic, the approach can differentiate top trending topics and persons in different geographical regions. Fourth, we constrain a timezone to a set of possible countries and use language features for training a high-level geocoder to further localize an influencer's geographic area. Finally, we provide a repository of geo-influencers for applications related to content recommendation. The repository can be used for filtering influencers based on their audience's demographics related to location, time, language, gender, and ethnicity.
